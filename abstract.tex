The de facto compilation model for production software compiles all
modules of a target program with a single set of compilation flags,
typically \emph{O2} or \emph{O3}. Such a per-program compilation
strategy may yield sub-optimal executables since programs tend to have
multiple hot loops with diverse code structures and can be further
optimized using targeted compilation flags. An alternative to the
per-program compilation model is a per-region compilation model that
assembles an optimized executable by combining the best per-region code variants.

In this paper, we demonstrate that a na\"ive greedy approach to per-region
compilation often degrades performance in comparison to the \emph{O3} baseline.
To overcome this problem, we contribute a novel per-loop compilation framework,
\toolname, which employs light-weight profiling to collect per-loop
timing information, and then utilizes a space-focusing technique to construct a
performant executable. Experimental results show that \toolname can reliably
improve performance of modern scientific applications on several
multi-core architectures by 10.3\% and 5.6\% (geometric mean) in
comparison to the \emph{O3} baseline and prior work, respectively.